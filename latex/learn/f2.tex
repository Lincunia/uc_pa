\documentclass{article}

\usepackage{graphicx}
\usepackage{amssymb, amsmath} %Paquetes matemáticos de la American Mathematical Society

\begin{document}
  \begin{titlepage}
    \centering
    {\includegraphics[width=0.2\textwidth]{../../pictures/uc.png}\par}
    \vspace{1cm}
    {\bfseries\LARGE Universidad Central\par}
    \vspace{1cm}
    {\scshape\Large Facultad de Ingenier\'ia de la que fuera \par}
    \vspace{3cm}
    {\scshape\Huge Intento pretensioso de aprender \LaTeX \par}
    \vspace{3cm}
    {\itshape\Large Proyecto Fin de Carrera \par}
    \vfill
    {\Large Autor: \par}
    {\Large Un estudiante \par}
    \vfill
    {\Large \today \par}
  \end{titlepage}

  \begin{equation*}
    F = ma \quad\text{Segunda ley de Newton} % El quad es para crear un espacio entre la ecuación y el texto
    % TAmbién está \textit para texto en cursiva, \textbf para texto en negrita y \textsf para tipo de letra sans serjf.
  \end{equation*}

  \begin{align*}
    ((a+b)^2 = a^2 + 2ab + b^2\\
    (a-b)^2 = a^2 - 2ab + b^2\\
    (a+b)(a-b) = a^2 - b^2
  \end{align*}
  % lo mismo que el align pero sin números
  \begin{equation*}
    \begin{split}
      (a+b)^2& = (a+b)(a+b)\\
      & = a^2 + ab + ab + b^2\\
      & = a^2 + 2ab + b^2
    \end{split}
  \end{equation*}


  {\bfseries\LARGE Prueba para escribir la parte lógica del taller \par}

  \begin{equation*}
    \begin{split}
      p: \quad\text{“El paro continúa”} \\
      q: \quad\text{”Nos quedaremos en casa a limpiar”} \\
      r: \quad\text{”Nos quedaremos en casa a dormir”} \\
      s: \quad\text{”No iremos mañana a la fiesta”} \\\\
    \end{split}
  \end{equation*}
  \begin{equation*}
    \begin{split}
      \left( p \lor q \right) \implies \left( r \lor s \right) \\
      \left( r \lor s \right) \implies \neg t \\
      t \\
      \noindent\rule{5cm}{0.4pt} \\
      p \\\\
    \end{split}
  \end{equation*}
  \begin{align}
    \left( p \lor q \right) \implies \left( r \lor s \right) \quad\text{ \indent  Premisa 1} \\
    \left( r \lor s \right) \implies \neg t \quad\text{  \indent Premisa 2} \\
    t \quad\text{ \indent Premisa 3} \\
    \left(p \lor q \right) \implies \neg t \quad\text{  \indent SH(1, 2)} \\
    \neg \left( p \lor q \right) \quad\text{ \indent MTT(3, 4)} \\
    \neg p \land \neg q \quad\text{ \indent Ley de Morgan (5)} \\
    \neg p \quad\text{ \indent RS (6)} \\
    \neg q \quad\text{ \indent RS (6)}
  \end{align}
  Conclusión: No es válido el argumento
\end{document}
