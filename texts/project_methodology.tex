\documentclass{article}
\usepackage{geometry, titlesec} % titlesec is to build a table of contents based on the sections
\geometry{
	letterpaper,
	left=25mm,
	top=25mm
}

\title{Methodology – Engineering practices I}
\author{Jorge Eliecer Vargas Puerto}
\date{September 18, 2023}

\begin{document}

\maketitle

\tableofcontents

\section{DESCRIPTION OF THE PROBLEM}
\par Characterizing a problem to propose a solution is a process that consists
of Identify, describe and analyze the problem clearly and precisely, in order to
generate and Evaluate possible solutions. Here the background is established and
the Questions Problems

\subsection{Identify the problem:}
\par It is about recognizing the existence of a problematic situation, which
implies a difficulty or a challenge, and that requires a solution.

\par To identify the problem, we can ask questions such as:

\begin{itemize}
	\item What does it do? - How does it do it?
	\item If there is a problem, what causes it?
	\item What's going on?
	\item What should be going on?
	\item Which are the needs of the Current System?
	\item What are the consequences of the problem?
	\item How often does the problem occur?
	\item Who is involved or affected by the problem?
	\item How large is the transaction volume?
	\item Number of Users accessing the system?
	\item How efficiently are tasks performed?
\end{itemize}

\subsection{Describing the problem:}
\par It is about defining the problem clearly and precisely, Using objective
data and avoiding value judgments or premature solutions Establishes a narrative
description of the problem The functional requirements in order to optimize the
Information System.

\par To describe the problem, we can ask questions such as:

\begin{itemize}
	\item What traits does the problem have?
	\item What evidence is there of the problem?
	\item How can the problem be measured or quantified?
\end{itemize}

\subsection{Analyzing the problem:}
\par It is about exploring the causes and effects of the problem, as well as
the factors that influence or condition it. To analyze the problem, you can ask
questions such as:

\begin{itemize}
	\item Why does the problem occur?
	\item What factors favor or hinder the issue?
	\item What impact does the issue have on the context?
\end{itemize}

\par Here the Current System is analyzed, and the needs are identified, It is
about recognize the existence of a problematic situation, which involves a
difficulty or a challenge, and one that requires a solution.

\par An Investigation is established, when identifying the problem that is the
starting point of the investigation, arises when conflicts are found in the
solution of the problem, within a set of known data, or a problem fact is an
event that does not fit within the expectations of the field of study and the
need arises to optimize the System

\par In the Identification of the problem you must define the opportunity, which
can present the Computer Science in using tools that minimize and optimize the
process:

\par There are tools and techniques that support the identification of
requirements and needs such as the Interview, the Survey, the Questionnaire, the
Observation, a Process Map. All these instruments must be applied at the time of
carrying out the study of each area of the Company, in order to look for
weaknesses and strengths in the current technique and operation.

\begin{itemize}
	\item
	      \textbf{Interview:}
	      \par The interview is a face-to-face exchange of information. It is a
	      communication channel between the one who builds the model and the User.
	\item

	      \textbf{Survey:}
	      \par In a "survey" you must collect information from a "sample". This is
	      usually only a portion according to the requirement you want to meet.
	\item

	      \textbf{Questionnaire:}
	      \par Questionnaires provide a very useful alternative to conduct the
	      interview; However, there are certain characteristics that may be
	      appropriate in some situations and inappropriate in another.

	\item
	      \textbf{Observation:}
	      \par Another useful technique to define the project is the observation
	      of the tasks, activities and processes that people perform in order to
	      evaluate the real behavior of the information and establish the
	      identification of the causes that affect the final results.

	\item
	      \textbf{Process map:}
	      \par It begins with a corporate strategic plan, with the aim of knowing
	      better and more deeply the operation and performance of the processes
	      and activities in which it is involved, paying special attention to
	      those key aspects of these.
\end{itemize}

\par "Cause and Effect Diagrams" are visual tools used to analyze and represent
the possible causes of a particular problem or effect. These diagrams are
especially useful for identifying the root causes of a problem and understanding
how different factors can contribute to an outcome.

\subsection{Problem wording:}
\par The problem wording is a crucial stage in the beginning of any project. It
consists of clearly defining the current situation or condition that the
project seeks to improve or solve. It specifies in a detailed and precise way
what is the problem that will be addressed through the project. Here I explain
in more detail its importance and how to carry it out:

\subsubsection{Prominence of the problem wording}
\begin{itemize}
	\item
	      \textbf{Clarify the Purpose of the Project:}
	      \par It helps to understand what is the fundamental purpose of carrying
	      out the project.
	\item
	      \textbf{Establish the Focus y Scope:}
	      \par Define which aspects of the problem will be addressed and which
	      will not.
	\item
	      \textbf{Guides Subsequent Decisions:}
	      \par It provides guidance for decision-making in later stages of the
	      project.
	\item
	      \textbf{Facilitates Communication:}
	      \par It enables clear and effective communication between team members
	      and stakeholders.
	\item
	      \textbf{Facilitates the Evaluation of Results:}
	      \par By having a clear definition of the problem, it is possible to
	      effectively assess whether the project has achieved its goal.
\end{itemize}

\subsubsection{Problem wording steps:}
\begin{itemize}
	\item
	      \textbf{Identification of the issue:}
	      \par Identify the situation, condition or need that motivates the
	      realization of the project. It can be a lack, a challenge or an
	      opportunity for improvement.
	\item
	      \textbf{Clear and precise definition:}
	      \par Describe the problem in detail and specific. Avoid generalities or
	      ambiguities.
	\item
	      \textbf{Setting limits of the Problem:}
	      \par Sets the boundaries of the problem to avoid addressing aspects that
	      are outside the scope of the project.
	\item
	      \textbf{Contextualization:}
	      \par Place the problem in context, explaining why it is relevant and
	      what the implications are of not solving it.
	\item
	      \textbf{Cause and effect analysis:}
	      \par Identify the possible causes and effects of the problem. Understand
	      how it affects the parties involved.
	\item
	      \textbf{Solution terms wording:}
	      \par Formulate the problem in terms that will suggest possible
	      solutions. For example, instead of "Lack of efficiency in the
	      production process", you could say "Need to improve the efficiency of
	      the production process".
	\item
	      \textbf{Identification of the benefits and interested parts:}
	      \par Determine who will be affected by the solution of the problem and
	      who has an interest in solving it.

	\item
	      \textbf{Documentation and communication:}
	      \par Be sure to document the formulation of the problem and communicate
	      it clearly to everyone involved in the project. A well-defined problem
	      formulation lays the foundation for the development of a successful
	      project, as it provides an accurate understanding of what is sought to
	      be achieved and why it is important to address that specific issue.
\end{itemize}

\section{GOALS AND SCOPE ESTABLISHMENT}
\par The objectives of the automatic information system and its scope are
clearly defined, make sure that the objectives are aligned with the strategic
goals of the Organization.

\par The General Objective of the Proposed System is established, to determine
its scope, where it begins, what is the environment of its Operation.
Portability and scalability of processes.

\par With the above, the Specific Objectives are established which support the
General Objective.

\par Defining the general goal and the specific goals is a crucial part of
planning any project

\subsection{General goal:}
\par The overall objective is a broad and concise statement that describes the
fundamental purpose of the project. You need to answer the question: "What are
you hoping to achieve with this project as a whole?" For an automation project,
the overall goal might be:

\begin{quote}
	\textbf{Example for an Automation Project in a Food Products Factory:}
	\par Optimize operational efficiency and product quality by implementing
	automation systems for food production and packaging.

\end{quote}

\subsection{Specific goals:}
\par Specific objectives are concrete and detailed targets that contribute
directly to the achievement of the general goal. These must be measurable and
achievable. It is recommended to use the SMART methodology to define them
(Specific, Measurable, Achievable, Relevant and with a defined Time). The SMART
methodology is an acronym used to set clear and achievable goals. Each letter in
SMART represents a criterion that a goal must be accomplished:

\begin{enumerate}
	\item
	      \textbf{Specific:}
	      The objective should be clear and specific, avoiding ambiguities.
	\item
	      \textbf{Measurable:}
	      It should be possible to quantify or measure progress towards the goal.
	\item
	      \textbf{Achievable:}
	      The goal must be realistic and achievable with the resources and time
	      available.
	\item
	      \textbf{Relevant:}
	      It should be aligned with the broader objectives and goals of the
	      person or organization.
	\item
	      \textbf{Temporally defined:}
	      A deadline or deadline must be set to achieve the goal.
\end{enumerate}

\par This methodology is widely used in areas such as the management of
projects, strategic planning and personal development, as it provides a Clear
and effective structure to set and achieve goals.

\end{document}
