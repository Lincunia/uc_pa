% Esto es una función en latex, y en ella se le pueden meter determinados
% argumentos; en este caso lo que tenemos es una función que define el formato
% del documento
\documentclass{article}

% maketitle begin
\title{Trying to learn {\LaTeX}}
\date{\today}
\author{A dumb student}
% maketitle end 

% La función con la que se pueden usar herramientas gráficas
\usepackage{graphicx}

\begin{document}

\maketitle

\section{First title}

Lorem ipsum dolor sit amet, officia excepteur ex fugiat reprehenderit enim
labore culpa sint ad nisi Lorem pariatur mollit ex esse exercitation amet. Nisi
anim cupidatat excepteur officia. Reprehenderit nostrud nostrud ipsum Lorem est
aliquip amet voluptate voluptate dolor minim nulla est proident. Nostrud
officia pariatur ut officia. Sit irure elit esse ea nulla sunt ex occaecat
reprehenderit commodo officia dolor Lorem duis laboris cupidatat officia
voluptate. Culpa proident adipisicing id nulla nisi laboris ex in Lorem sunt
duis officia eiusmod. Aliqua reprehenderit commodo ex non excepteur duis sunt
velit enim. Voluptate laboris sint cupidatat ullamco ut ea consectetur et est
% Esto es para hacer el texto en negrita
\textbf{culpa et culpa duis.}

\subsection{The first subtitle}

% Aparentemente ambas funciones generan lo mismo, pero estilísticamente se
% pueden apreciar las diferencias
\textit{Lorem ipsum dolor sit amet, qui minim labore adipisicing minim sint}
\emph{cillum sint consectetur cupidatat.}

\subsubsection{How to underline things}

\underline{Sharing with your partner is good, but take into account the stuff
that you have} That's an important aspect of coexistance with different people.

\section{Second title (Picture time)}

Love is a lie girl, Love is a mess; elling these idiots of loneliness; our love
survive, our love's alive.

Letters are written, words are spoken; some hearts are better left unbroken; our
love survive, our love's alive.

Little by little, step by step; we were called the dirty pets; heart to heart
and side by side, love will never hide.

Locomotion tango on a lonely rainbow; we will dancing baby, don't wait for
heroes. Locomotion tango on a lonely rainbow; we will dancing baby, romancing
lady.

Locomotion in my heart; see you dancin' in the darkness. Locomotion let us
start, let us dancing in the moonlight

% No me pregunten por qué los comentarios los hago en español y el texto lo
% hago en inglés... ni yo mismo lo sé XD
At this point, look at those genious in the Figure~\ref{fig:lords} and follow
their rules

% Esto es para poder insertar las imagenes una vez se tenga el paquete graphicx
% y de ese modo se puede tener control de lo que se quiere insertar
\begin{figure}[ht]
  % \centering <- Figura centrada.
	% \raggedleft <- Figura alineada a la derecha.
	% \raggedright <- Figura alineada a la izquierda. 
  \begin{center}
    \includegraphics[width=4cm]{../../pictures/saint_ignucius.jpg}
    \includegraphics[width=4cm]{../../pictures/hippie_jesus.png}
  \end{center}
  % Texto en relación con las imágenes
  \caption{Wise people}
  % Referencias como figuras cada que se desean llamar
  \label{fig:lords}
\end{figure}

Lorem ipsum dolor sit amet, officia excepteur ex fugiat reprehenderit enim
labore culpa sint ad nisi Lorem pariatur mollit ex esse exercitation amet. Nisi
anim cupidatat excepteur officia. Reprehenderit nostrud nostrud ipsum Lorem est
aliquip amet voluptate voluptate dolor minim nulla est proident. Nostrud
officia pariatur ut officia. Sit irure elit esse ea nulla sunt ex occaecat
reprehenderit commodo officia dolor Lorem duis laboris cupidatat officia
voluptate. Culpa proident adipisicing id nulla nisi laboris ex in Lorem sunt
duis officia eiusmod. Aliqua reprehenderit commodo ex non excepteur duis sunt
velit enim. Voluptate laboris sint cupidatat ullamco ut ea consectetur et est

\end{document}

