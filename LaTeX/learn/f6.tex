\documentclass{article}
\usepackage{amssymb, amsmath}
\usepackage{geometry}
\geometry{
	letterpaper,
	left = 25mm,
	top = 25mm
}
\DeclareUnicodeCharacter{2212}{\ensuremath{-}}

\title{
	A supposed homework \\
	\large A task created by some students that didn't have something better to do
}
\author{A series of dumb students}
\date{\today}

\begin{document}

\maketitle

\section{Part 1}
\subsection{James Stewart Page 10  -  Exercises from 25 to 30}

\begin{enumerate}
	\item
	      \begin{enumerate}
		      \item
		            \begin{align*}
			            \frac{3}{10}  +  \frac{4}{15} = \frac{45 + 40}{150}
			            = \frac{85}{150}  = \frac{19}{30}
		            \end{align*}
		      \item
		            \[ \frac{1}{4}  +  \frac{1}{5}  = \frac{4  +  5}{20}  = \frac{9}{20} \]
	      \end{enumerate}

	\item
	      \begin{enumerate}
		      \item
		            \begin{align*}
			            \frac{2}{3} - \frac{3}{5} = \frac{10 - 9}{15}
			            = \frac{1}{15}
		            \end{align*}
		            \begin{align*}
		            \end{align*}
		      \item
		            \begin{align*}
			            1 + \frac{5}{8} - \frac{1}{6} = \frac{8 + 5}{8} - \frac{1}{6}
			            = \frac{70}{48}  = \frac{35}{24}
		            \end{align*}
		      \item
	      \end{enumerate}
	\item
	      \begin{enumerate}
		      \item
		            \begin{align*}
			            \frac{2}{3} * \left(6 - \frac{3}{2}\right) = \frac{2}{3} * \left( \frac{12 - 3}{2} \right)
			            = \frac{2}{3} * \frac{9}{2}
			            = \frac{18}{6}  = 3
		            \end{align*}
		      \item
		            \begin{align*}
			            0.25\left(\frac{8}{9} + \frac{1}{2}\right) = \frac{25}{100}\left(\frac{16 + 9}{18}\right)
			            = \frac{1}{4} * \frac{25}{18}  = \frac{25}{72}
		            \end{align*}
	      \end{enumerate}
	\item
	      \begin{enumerate}
		      \item
		            \begin{align*}
			            \left(3 + \frac{1}{4}\right)\left(1 - \frac{4}{5}\right)
			            = \left(\frac{12 + 1}{4}\right)\left(\frac{5 - 4}{5}\right)
			            = \frac{13}{20}\end{align*}
		      \item
		            \begin{align*}
			            \left(\frac{1}{2} - \frac{1}{3}\right)\left(\frac{1}{2} + \frac{1}{3}\right)
			            = \left(\frac{1}{6}\right)\left(\frac{5}{6}\right)
			            = \frac{5}{36}
		            \end{align*}
	      \end{enumerate}
	\item
	      \begin{enumerate}
		      \item
		            \begin{align*}
			            \frac{2}{\frac{2}{3}} - \frac{\frac{2}{3}}{2}
			            = \frac{\frac{4}{1} - \frac{4}{9}}{\frac{4}{3}}
			            = \frac{\frac{36 - 4}{9}}{\frac{4}{3}}
			            = \frac{96}{36}
			            = \frac{8}{3}
		            \end{align*}

		      \item
		            \begin{align*}
			            \frac{\frac{1}{12}}{\frac{1}{8} - \frac{1}{9}}
			            = \frac{\frac{1}{12}}{\frac{9 - 8}{72}}
			            = \frac{72}{12}
			            = 6
		            \end{align*}
	      \end{enumerate}
	\item
	      \begin{enumerate}
		      \item
		            \begin{align*}
			            \frac{2 - \frac{3}{4}}{\frac{1}{2} - \frac{1}{3}}
			            = \frac{\frac{8 - 3}{4}}{\frac{3 - 2}{6}}
			            = \frac{\frac{5}{4}}{\frac{1}{6}}
			            = \frac{30}{4}
			            = \frac{15}{2}
		            \end{align*}
		      \item
		            \begin{align*}
			            \frac{\frac{2}{5} + \frac{1}{2}}{\frac{1}{10} + \frac{3}{15}}
			            = \frac{\frac{4 - 5}{10}}{\frac{15 + 30}{150}}
			            = \frac{\frac{9}{10}}{\frac{45}{150}}
			            = \frac{1350}{450}
			            = 3
		            \end{align*}
	      \end{enumerate}

\end{enumerate}

\section{Part 2}
\subsection{James Stewart Page 21  -  Exercises from 15 to 24}

\begin{enumerate}
	\item
	      \begin{enumerate}
		      \item
		            \begin{align*}
			            -3^2  = -9
		            \end{align*}
		      \item
		            \begin{align*}
			            (-3)^2  = 9
		            \end{align*}
		      \item
		            \begin{align*}
			            \left(\frac{1}{3}\right)^4(-3)^2
			            = \frac{1}{81}*9
			            = \frac{1}{9}
		            \end{align*}
	      \end{enumerate}
	\item
	      \begin{enumerate}
		      \item
		            \begin{align*}
			            5^4*5^{-2}
			            = \frac{625}{25}
			            = 25
		            \end{align*}
		      \item
		            \begin{align*}
			            \frac{10^7}{10^4}
			            = 10^{7-4}
			            = 1000
		            \end{align*}
		      \item
		            \begin{align*}
			            \frac{3}{3^{-2}}
			            = \frac{\frac{3}{1}}{\frac{1}{9}}
			            = 27
		            \end{align*}
	      \end{enumerate}
	\item

	      \begin{enumerate}
		      \item
		            \begin{align*}
			            \left(\frac{5}{3}\right)^0*2^{-1}
			            = 1*\frac{1}{2}
			            = 0.5
		            \end{align*}
		      \item
		            \begin{align*}
			            \frac{2^{-3}}{3^0}
			            = \frac{1}{8}
		            \end{align*}
		      \item
		            \begin{align*}
			            \left(\frac{1}{4}\right)^{-2}
			            = \frac{1}{1/16}
			            = 16
		            \end{align*}
	      \end{enumerate}
	\item
	      \begin{enumerate}
		      \item
		            \begin{align*}
			            \left(-\frac{2}{3}\right)^{-3}
			            = \frac{1}{-\frac{8}{27}}
			            = -\frac{27}{8}
		            \end{align*}
		      \item
		            \begin{align*}
			            \left(\frac{3}{2}\right)^{-2}\left(\frac{9}{16}\right)
			            = \frac{4}{9}*\frac{9}{16}
			            = \frac{36}{144}
			            = \frac{1}{4}
		            \end{align*}
		      \item
		            \begin{align*}
			            \left(\frac{1}{2}\right)^4*\left(\frac{5}{2}\right)^{-2}
			            = \frac{1}{16}*\frac{4}{25}
			            = \frac{4}{400} = \frac{1}{100}
		            \end{align*}
	      \end{enumerate}
	\item
	      \begin{enumerate}
		      \item
		            \begin{align*}
			            \sqrt{16}
			            = 4
		            \end{align*}
		      \item
		            \begin{align*}
			            \sqrt[4]{16}
			            = 2
		            \end{align*}
		      \item
		            \begin{align*}
			            c. \sqrt[4]{\frac{1}{16}}
			            = \frac{1}{\sqrt[4]{16}}
			            = \frac{1}{2}
		            \end{align*}
	      \end{enumerate}
	\item
	      \begin{enumerate}
		      \item
		            \begin{align*}
			            \sqrt{64}
			            = 8
		            \end{align*}
		      \item
		            \begin{align*}
			            \sqrt[3]{-64}
			            = -4
		            \end{align*}
		      \item
		            \begin{align*}
			            c. \sqrt[5]{-32}
			            = -2
		            \end{align*}
	      \end{enumerate}
	\item
	      \begin{enumerate}
		      \item
		            \begin{align*}
			            \sqrt{\frac{4}{9}} = \frac{2}{3}
		            \end{align*}
		      \item
		            \begin{align*}
			            \sqrt[4]{256} = 4
		            \end{align*}
		      \item
		            \begin{align*}
			            \sqrt[6]{\frac{1}{64}}
			            = \frac{1}{\sqrt[6]{64}}
			            = \frac{1}{2}
		            \end{align*}
	      \end{enumerate}
	\item
	      \begin{enumerate}
		      \item
		            \begin{align*}
			            \sqrt{7}\sqrt{28}
			            = \sqrt{196}
			            = 14
		            \end{align*}
		      \item
		            \begin{align*}
			            \frac{\sqrt{48}}{\sqrt{3}}
			            = \frac{4\sqrt{3}}{\sqrt{3}}
			            = 4
		            \end{align*}
		      \item
		            \begin{align*}
			            \sqrt[4]{24}*\sqrt[4]{54}
			            = \sqrt[4]{1296}
			            = 6
		            \end{align*}
	      \end{enumerate}
	\item
	      \begin{enumerate}
		      \item
		            \begin{align*}
			            \left(\frac{4}{9}\right)^{-\frac{1}{2}}
			            = \frac{1}{\sqrt{\frac{4}{9}}}
			            = \frac{1}{\frac{2}{3}}
			            = \frac{3}{2}
		            \end{align*}
		      \item
		            \begin{align*}
			            (-32)^{\frac{2}{5}}
			            = \left[\sqrt[5]{-32}\right]^2
			            = \sqrt[5]{1024}
			            = 4
		            \end{align*}
		      \item
		            \begin{align*}
			            -32^{2/5}
			            = \sqrt[5]{-32^2}
			            = \sqrt[5]{-1024}
			            = 4
		            \end{align*}
	      \end{enumerate}
	\item
	      \begin{enumerate}
		      \item
		            \begin{align*}
			            1024^{-0.1}  = 1024^{-\frac{1}{10}}
			            = \frac{1}{\sqrt[10]{1024}}
			            = \frac{1}{2}
		            \end{align*}
		      \item
		            \begin{align*}
			            \left(-\frac{27}{8}\right)^{2/3}
			            = \sqrt[3]{\left(\frac{729}{64}\right)}
			            = \frac{\sqrt[3]{729}}{\sqrt[3]{64}}
			            = \frac{9}{4}
		            \end{align*}
		      \item
		            \begin{align*}
			            \left(\frac{25}{64}\right)^{\frac{3}{2}}
			            = \frac{1}{\left(\frac{25}{64}\right)^{\frac{3}{2}}}
			            = \frac{1}{\left(\sqrt{\frac{{25^3}}{64^3}}\right)}
			            = \frac{1}{\frac{125}{512}}
			            = \frac{512}{125}
		            \end{align*}
	      \end{enumerate}
\end{enumerate}

\section{Part 3}
\subsection{Earl W. Swokowski: Page 376 - Exercises 11 to 16}

\begin{enumerate}
	\item
	      \begin{align}
		      2\log_ax+\frac{1}{3}\log_a(x-2)-5\log_a(2x+3)   \\
		      =\log_ax^2+\log_a\sqrt[3]{(x-2)}-\log_a(2x+3)^5 \\
		      =\log_a(x^2\sqrt[3]{(x-2)})-\log_a(2x+3)^5      \\
		      =\log_a\left[\frac{x^2\sqrt[3]{(x-2)}}{(2x+3)^5}\right]
	      \end{align}
	\item
	      \begin{align}
		      5\log_ax-\frac{1}{2}\log_a(3x-4)-3\log_a(5x+1)                  \\
		      =\log_ax^5-\log_a\sqrt[2]{(3x-4)}-\log_a(5x+1)^3                \\
		      =\log_a\left(\frac{x^5}{\sqrt[2]{(3x-4)}}\right)-\log_a(5x+1)^3 \\
		      =\log_a\left(\frac{\frac{x^5}{\sqrt[2]{(3x-4)}}}{(5x+1)^3}\right)
		      =\log_a\left(\frac{x^5}{\sqrt[2]{(3x-4)}*(5x+1)^3}\right)
	      \end{align}
	\item
	      \begin{align}
		      \log (x^3y^2)-2\log x\sqrt[3]{y}-3\log (x/y)                     \\
		      =\log\left[\frac{(x^3y^2)}{(x\sqrt[3]{y})^2}\right]-\log (x/y)^3 \\
		      =\log\left[\frac{\frac{(x^3y^2)}{(x\sqrt[3]{y})^2}}{(x/y)^3}\right]
		      =\log\left[\frac{x^3y^2}{(x\sqrt[3]{y})^2(x/y)^3}\right]
	      \end{align}
	\item
	      \begin{align}
		      2\log\left(\frac{y^3}{x}\right)-3\log y+\frac{1}{2}\log x^4y^2           \\
		      = \log\left[\frac{(\frac{y^3}{x})^2}{4^3}\right] +\frac{1}{2}\log x^4y^2 \\
		      = \log\left[\frac{(\frac{y^3}{x})^2*\sqrt{(x^4y^2)}}{4^3}\right]
	      \end{align}
	\item
	      \begin{align}
		      \ln y^3+\frac{1}{3}\ln (x^3y^6)-5\ln y           \\
		      = \ln \left(y^3\sqrt[3]{(x^3y^6)}\right)-\ln y^5 \\
		      = \ln \left(\frac{y^3\sqrt[3]{(x^3y^6)}}{y^5}\right)
	      \end{align}
	\item
	      \begin{align}
		      2\ln x-4\ln \frac{1}{y}-3\ln (xy)                   \\
		      = \ln x^2-\ln \left(\frac{1}{y}\right)^4-\ln (xy)^3 \\
		      = \ln \left[\frac{\frac{x^2}{\frac{1}{y^4}}}{(xy)^3}\right]
		      = \ln \left[\frac{x^2}{\frac{1}{y^4}*(xy)^3}\right]
	      \end{align}
\end{enumerate}

\section{Part 4}
\subsection{James Stewart: Page 43 - Exercises pairs from 2 to 20}

\begin{enumerate}
	\item
	      \begin{align}
		      (7x^3+2x^2-11x)+(-3x^3-2x^2+5x-3) = 4x^3-6x-3
	      \end{align}
	\item
	      \begin{align}
		      (6x^3-2x^2+x-2)-(8x^2-x-2) = 6x^3-10x^2+2x
	      \end{align}
	\item
	      \begin{align}
		      (3x-4)(2x+9) & = 6x^2+27x-8x-36 \\
		                   & = 6x^2+19x-36
	      \end{align}
	\item
	      \begin{align}
		      (4x-3y)(x-5y)         \\
		      = 4x^2-20xy-3xy-15y^2 \\
		      =4x^2-23xy-15y^2
	      \end{align}
	\item
	      \begin{align}
		      (3u-1)(u+2)+7u(u+1) & = 3u^2+6u-u-2+7u^2+7u \\
		                          & = 10u^2+12u-2
	      \end{align}
	\item
	      \begin{align}
		      (7x-4)(x^3-x^2+6)            \\
		      = 7x^4-7x^3+42x-4x^3+4x^2-24 \\
		      =7x^4-11x^3+4x^2+42x-24
	      \end{align}
	\item
	      \begin{align}
		      (r^2-8r-2)(-r^2+3r-1)                   \\
		      = -r^4+3r^3-r^2+8r^3-24r^2-8r+2r^2-6r+2 \\
		      = -r^4+11r^3-23r^2-14r+2
	      \end{align}
	\item
	      \begin{align}
		      (2x-1)(x^2-5)(x^3-1)                 \\
		      = (2x^3-10x-x^2+5)(x^3-1)            \\
		      = 2x^6-10x^4-x^5+5x^3-2x^3+10x+x^2-5 \\
		      = 2x^6-x^5-10x^4+3x^3+x^2+10x-5
	      \end{align}
	\item
	      \[ \frac{6a^3b^3-9a^2b^2+3ab^4}{3ab^2} = 2a^2b-3a+b^2 \]
	\item
	      \[ \frac{6x^2yz^3-xy^2z}{xyz}=6xz^2-y \]
\end{enumerate}

\section{Part 5}
\subsection{Earl W. Swokowski: Page 43 - Exercises about multiples of 3 from 45 to 102}

\begin{enumerate}
	\item
	      \[ rs+4st=s[r+4t] \]
	\item
	      \[ 10xy+15xy^2=5xy[2+3y] \]
	\item
	      \[ 15x^3y^5-25x^4y^2+10x^6y^4=5x^3y^2[3y^2-5x+2x^3y^2] \]
	\item
	      \begin{align}
		      7x^2+10x-8=\frac{(7x)^2+10(7x)-56}{7} \\
		      =\frac{[7x+14][7x-4]}{7}=[x+2][7x-4]
	      \end{align}
	\item
	      \begin{align}
		      6x^2+7x-20=\frac{(6x)^2+7(6x)-120}{6} \\
		      =\frac{[6x+15][6x-8]}{6}=[2x+5][3x-4]
	      \end{align}
	\item
	      \begin{align}
		      21x^2+41x+10 = \frac{(21x)^2+41(21x)-210}{21} \\
		      =\frac{[21x+35][21x+6]}{21}=[3x+5][7x+2]
	      \end{align}
	\item
	      \[ 25x^2+30x+9 = [5+3]^2 \]
	\item
	      \begin{align}
		      50x^2 + 45xy - 18y^2 = [50x^2 - 15xy] + [60xy - 18y^2] \\
		      5x[10x - 3y] + 6y[10x - 3y] = [10x - 3y][5x + 6y]
	      \end{align}
	\item
	      \[ z^4 - 64w^2 = (z^2 - 8w)(z^2 + 8w) \]
	\item
	      \[ x^3 - 25x = x(x^2 - 25)=x(x + 5)(x - 5) \]
	\item
	      \[ 75x^2 - 48y^2=3(25x^2 - 16y^2)=3(5x - 4y)(5x + 4y) \]
	\item
	      \[ 125x^3 - 8=(5x - 2)(25x^2 + 10x + 4) \]
	\item
	      \[ 343x^3 + y^9=(7x + y^3)(49x^2 - 7xy^3 + y^6) \]
	\item
	      \[ x^3 + 64=(x + 4)(x^2 + 4x + 16) \]
	\item
	      \[ 2ax - 6bx + ay - 3by=2x(a - 3b) + y(a - 3b)=(2x + y)(a - 3b) \]
	\item
	      \[ x^4 - 3x^3 + 8x - 24=x^3(x - 3) + 8(x - 3)=(x^3 + 8)(x - 3) \]
	\item
	      \begin{align}
		      a^6 - b^6 & = (a^2 - b^2)(a^4  +  a^2 b^2  +  b^4)                   \\
		                & = (a  +  b)(a - b)(a^2  +  ab  +  b^2)(a^2 − ab  +  b^2)
	      \end{align}
	\item
	      \[ x^2 - 4y^2 - 6x + 9 = (x - 3)^2 - 4x^2 = (x−3 + 2y)(x−3−2y) \]
	\item
	      \begin{align}
		      a^6 - b^6 & = (a^2 - b^2)(a^4 + a^2b^2 + b^4)              \\
		                & = (a + b)(a - b)(a^2 + ab + b^2)(a^2−ab + b^2)
	      \end{align}
	\item
	      \begin{align}
		      y^6 + 7y^3 - 8 & = (y^3 - 1)(y^3 + 8)                   \\
		                     & =(y−1)(y^2 + y + 1)(y + 2)(y^2−2y + 4)
	      \end{align}
	\item
	      \begin{align}
		      4x^3 + 4x^2 + x & = x(4x^2 + 2x + 2x + 1)     \\
		                      & = x(2x(2x + 1) + 1(2x + 1)) \\
		                      & = x(2x + 1)^2
	      \end{align}
\end{enumerate}

\section{Part 6}
\subsection{Earl W. Swokowski: Page 91 - Exercises pairs from 1 to 20}

\begin{enumerate}
	\item
	      \begin{align}
		      4x^2 + x - 14 = 0
		      \implies \frac{(4x)^2 + (4x) - 56}{4} \\
		      = (x + 2)(4x - 7)                     \\
		      x + 2 = 0 \implies x =  - 2           \\
		      4x - 7 = 0 \implies x = \frac{7}{4}
	      \end{align}
	\item
	      \begin{align}
		      15x^2 - 14 = 29x \implies 15x^2 - 29x - 14 = 0 \\
		      \frac{ - ( - 29) \pm \sqrt{( - 29)^2 - 4*15* - 14}}{30}
		      = \frac{29 \pm \sqrt{1681}}{30}                \\
		      \frac{ - 12}{30} = \frac{ - 2}{5} = x          \\
		      \frac{70}{30} = \frac{7}{3} = x
	      \end{align}
	\item
	      \begin{align}
		      x(3x + 10) = 77 \implies 3x^2 + 10x - 77 = 0 \\
		      \implies \frac{(3x)^2 + 10(3x) - 231}{3}
		      = (x + 7)(3x - 11)                           \\
		      x + 7 = 0 \implies x =  - 7                  \\
		      3x - 11 = 0 \implies x = \frac{11}{3}
	      \end{align}
	\item
	      \begin{align}
		      48x^2 + 12x - 90 = 0             \\
		      \frac{ - (12) \pm \sqrt{(12)^2 - 4*48* - 90}}{96}
		      = \frac{ - 12 \pm 132}{96}       \\
		      \frac{120}{96} = \frac{5}{4} = x \\
		      \frac{ - 144}{96} = \frac{ - 3}{2} = x
	      \end{align}
	\item
	      \begin{align}
		      4x^2 - 72x + 324 = 0
		      \implies \frac{(4x)^2 - 72(4x) + 1296}{4}         \\
		      = (2x - 18)^2 \implies 2x - 18 = 0 \implies x = 9 \\
	      \end{align}
	\item
	      \begin{align}
		      \frac{5x}{x - 2} + \frac{3}{x} + 2 = \frac{ - 6}{x^2 - 2x} \\
		      \implies \frac{5x^2 + 3x - 6}{x^2 - 2x} + \frac{2(x^2 - x)}{x^2 - 2x}
		      = \frac{ - 6}{x^2 - 2x}                                    \\
		      \implies 5x^2 + 3x - 6 - 2x^2 - 4x = 0
		      \implies 7x^2 - x = 0                                      \\
		      \implies x(7x - 1) = 0                                     \\\\
		      x = 0                                                      \\
		      x = \frac{1}{7}
	      \end{align}
	\item
	      \begin{align}
		      \frac{3x}{x - 2} + \frac{1}{x + 2} = \frac{ - 4}{x^2 - 4}                         \\
		      \implies \frac{3x(x + 2)}{x^2 - 4} + \frac{x - 2}{x^2 - 4} = \frac{ - 4}{x^2 - 4} \\
		      \implies 3x^2 + 7x - 2 =  - 4
		      \implies 3x^2 + 7x + 2                                                            \\
		      \frac{ - (7) \pm \sqrt{(7)^2 - 4*3*2}}{6}
		      = \frac{ - 7 \pm 5}{6}                                                            \\
		      \frac{ - 2}{6} = \frac{ - 1}{3} = x                                               \\
		      \frac{ - 12}{6} =  - 2 = x
	      \end{align}
	\item
	      \begin{enumerate}
		      \item
		            \begin{align}
			            x^2 = 25,x = 5                      \\
			            \sqrt{x^2} = \sqrt{25} = 5
			            \implies (x = 5)\thicksim(x^2 = 25) \\\\
		            \end{align}
		      \item
		            \begin{align}
			            x = \sqrt{64},x = 8 \\
			            \sqrt{64} = 8
			            \implies (x = 8) = (x = \sqrt{64})
		            \end{align}
	      \end{enumerate}
	\item
	      \begin{align}
		      x^2 = 361 \implies x^2 - 361 = 0 \\
		      \frac{ - (0) \pm \sqrt{(0)^2 - 4* - 361}}{2}
		      = \frac{\pm \sqrt{1444}}{2}      \\
		      \frac{38}{2} = 19 = x            \\
		      \frac{ - 38}{2} =  - 19 = x
	      \end{align}
	\item
	      \begin{align}
		      16x^2 = 49 \implies 16x^2 - 49 = 0 \\
		      \frac{ - (0) \pm \sqrt{(0)^2 - 4*16* - 49}}{32}
		      = \frac{\pm \sqrt{3136}}{32}       \\
		      \frac{56}{32} = \frac{7}{4} = x    \\
		      \frac{ - 56}{32} = \frac{ - 7}{4} = x
	      \end{align}
\end{enumerate}

\end{document}
