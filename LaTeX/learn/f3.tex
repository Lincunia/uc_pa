\documentclass{article}
\usepackage{amssymb, amsmath} % Mathematical packages from the American
% Mathematical Society, can be used for other things beyond integrals
\usepackage{geometry} % A way to manage the size of a document
\geometry{
	letterpaper,
	left=25mm,
	top=25mm
}

\begin{document}
% The most basic thing of doing integrals in LaTeX
\section*{The weirdest integrals}

\noindent Letter d \footnote{This can be determined by the product of the derivative and its differential in letter x} is a differential or growth. Thus, this situation occurs: $ \frac{\mathrm{d}F}{\mathrm{d}x} = f \Rightarrow \mathrm{d}F = f\mathrm{d}x $

\[ \mathrm{d}(e^x) = (e^x)\mathrm{d}x \]

\[ x^{10} - x^{10} = 10x^9(x_f - x_i) \]

\noindent And now, here we got the new conceptions about \LaTeX, because we have got to do quotations of whatever is happening, and we're trying to implement that on our texts. All we ever wanted for biographies is here in our arms:

\begin{quote}
	The integral (summation) of the cosine of x is equal to the sine function of x
	with the addition of a cost.
\end{quote}

\[ \int \cos  x \mathrm{d}x = \sin  x + c \]

\noindent Here we got this exercise $ \int 5x \mathrm{d}x = \frac{5x^2}{2} + c $ which is an example of this quotation very important to explain the anti-derivatives:

\begin{quote}
	\[ \int f(x) \mathrm{d}x = F(x) + c \]
\end{quote}

% I don't now why this thing is like so

\[ \int e^{x^2}2x \mathrm{d}x = e^(x^2) + x \]

\[ \int \frac{1}{x^2 + 1}2x \mathrm{d}x = Ln(x^2 + 1) + c \]

\[ \int x^n \mathrm{d}x = \frac{x^{n + 1}}{n + 1} + c \]

\[ \int \tan x \mathrm{d}x = Ln|\sec x| + c \]

%%\par Esto para cuestiones como la cadena, se emplean este tipo de cosas:
\noindent This is for matters like the string, it's used in this example:

\begin{gather*}
	\int u^{10} \frac{\mathrm{d}u}{3} = \frac{1}{3} \int u^{10} \mathrm{d}u \\
	\frac{1}{3}*\frac{u^{11}}{11}+c = \frac{u^{11}}{33}+c
\end{gather*}

\noindent A rare exercise: % Mi yo del futuro estará decepcionado

\begin{gather*}
	\int \cos (5x) \mathrm{d}x \\
	u = 5x \Rightarrow \mathrm{d}u = 5\mathrm{d}x \\
	\int \cos (u) \frac{\mathrm{d}u}{5} = \frac{1}{5} \int \cos u \mathrm{d}u \\
	\frac{1}{5}\sin u + c = \frac{1}{5}\sin (5x) + c
\end{gather*}

\par Here we got another exercise to practice

\begin{gather*}
	\int \sqrt{x^5 + x}(5x^4 + 1) \mathrm{d}x \\
	u = x^5 + x \Rightarrow \frac{\mathrm{d}u}{\mathrm{d}x} = 5x^4 + 1 \\
	\int \sqrt{u} \mathrm{d}u = \frac{2}{3}*u*\sqrt{u} + c
\end{gather*}

\section*{The main learn}

\noindent The first step is separate the big trigonometric functions into little pieces
that let you put different thing that make you solve the problem.

\noindent The $\int \mathrm{d}u$ is equal to $u$ and you can proof that by using the derivative

\begin{gather*}
	\int \cos^3 x \mathrm{d}x \\
	\int \cos x*\cos^2 x \mathrm{d}x \\
	\int \cos x*(1-\sin^2 x) \mathrm{d}x \\
	u = \sin x \Rightarrow \mathrm{d}u = \cos x \mathrm{d}x \\
	\mathrm{d}x = \frac{\mathrm{d}u}{\cos x} \\
	\int \cos x*(1-u^2)\frac{\mathrm{d}u}{\cos x} \\
	\int (1 - u^2)\mathrm{d}u \Rightarrow u - \frac{u^3}{3} + c
\end{gather*}

\noindent Another exercise:

\begin{gather*}
	\int \cos^7 x \mathrm{d}x \\
	\int \cos^5 x \cos^2 x \mathrm{d}x \\
	u = \sin x \Rightarrow \mathrm{d}x = \frac{\mathrm{d}u}{\cos x} \\ \\
	\int (1 - u^2)^3 \mathrm{d}u \\
	\int (1 - 3u^2 + 3u^4 - u^6) \mathrm{d}u \\
	u - u^3 + \frac{3 * u^5}{5} - \frac{u^7}{7} + c \\
	\sin x - \sin^3 x + \frac{3*\sin^5 x}{5} - \frac{\sin^7 x}{7} + c \\
\end{gather*}

\noindent Now's the time where we must solve this exercise (I got difficulties in \LaTeX) by
writing it: $ \int_0^\pi \sin^2 x \mathrm{d}x $ and I don't know how to get through

\end{document}
