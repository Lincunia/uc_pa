\documentclass{article}

\usepackage{amssymb, amsmath}
\usepackage{amsfonts}
\newcommand{\field}[1]{\mathbb{#1}}
\newcommand{\C}{\field{C}}
\newcommand{\R}{\field{R}}

\begin{document}
El cambio de coordenadas hacen que aparezca un factor que modifique nuestro
sistema de ecuaciones. Dependiendo de la función de la que hicieramos de la
integral. Dependeiendo de la fomra ne cómo quisieramos integrar.

Con base a ese hecho, se tiene que hablar de lo que es un cambio de coordenadas
$  F: \R \longrightarrow \R^{n} $ Y no se puede obtener una función inversa
sabiendo que $ r $ es otra cosa. Para los valores de $\tan(x)$ se tienen
diversos puntos. En general, tenemos este conjunto:
\begin{align}
  S = \{(r, \theta)\R^2: r>0, 0 \leq \theta \leq 2 \pi\} \\
  T(r, \theta) = (r\cos{\theta}, r\sin{\theta})
\end{align}

\section*{Un ejemplo}

Sea $ T: \R^2 \longrightarrow \R^2 $ una transformación lineal dada por:
\begin{align}
  T(u, v) = (2u+v. u-v) \\
  T(u, v) = \left( \begin{matrix}
    2 & 1 \\
    1 & -1
  \end{matrix} \right)
  \left( \begin{matrix}
    u\\
    v
  \end{matrix} \right) = \left( \begin{matrix}
    u \\
    v
  \end{matrix} \right)
\end{align}

Note que $ det(A)\neq0$ luego $T$ tiene inversa. Además $T$ tiene derivadas
parciales continuas. $T$ es un cambio de coordenadas sabiendo que
$ T:S \longrightarrow R $ por lo que
$ S = \{(u, v) \in \R^2 | 0 \leq u \leq 1, -1 \leq v \leq 1\}$
$ T = \{(x, y) \in \R^2 | 0 \leq u \leq 1, -1 \leq v \leq 1\}$

\section*{Matriz jacobiana}
Aplicable para transformaciones lineales como se venía viendo desde la sección
previa.

\begin{align}
  T(u, v) = (2u+v. u-v)  =  (x, y) \\
  T_r(u, v) = \left( \begin{matrix}
    1 & -2 \\
    1 & 1
  \end{matrix} \right) \\
  det(T_r) &= -2 -1 \\
  &= -3 \\
\end{align}

\begin{align}
  T(\rho, \theta, \phi) = e^{\rho^2} * \rho^2 * \sin{\phi} \\
  T_r(u, v) = \left( \begin{matrix}
    \cos\theta{}\sin{\phi} & -\rho\sin{ \theta}\cos{\phi} & \rho\cos{\theta}\cos{\phi} \\
    \sin{\theta}\sin{\phi} & \rho\cos{\theta}\sin{\phi} & \rho\sin{\theta}\cos{\phi} \\
    \cos{\phi} & 0 & -\rho\sin{\phi} \\
  \end{matrix} \right) \\
  deg(T_r(\rho, \theta, \phi)) = -\rho^2\cos{\theta}^2sin{\phi}^3 - \rho^2\sin{\theta}^2\cos{\phi}^2 - \rho^2\cos{\theta}^2\cos^2{\theta} - \rho^2\sin{\theta}^2\sin{\phi}^3 \\
  deg(T_r(\rho, \theta, \phi)) = -\rho^2(\sin{\phi}^3+\cos{\phi}^2\sin{\phi})
\end{align}

\section*{Integral indefinida para funciones vectoriales}

La derivada de una función vectorial indica la velocidad de ese objeto en el que
se desplaza en el punto concreto en el espacio.

Una partícula tiene una velocidad dada por la siguiente expresión:
$ r'(t) = (e^t-1, 2t) $ Determine la posición de la partícula cuando $ t = 3 $
si la posición inicial es $ (1, 0) $.

La posición 
\begin{align}
  r(t) &= \int r'(t) \mathrm{d}x \\
  &= (e^t-t+C_1, t^2+C_2)
\end{align}
Ahora, para cuando $ r(0) = (1, 0) $, se obtiene por lo tanto:
\begin{align}
  e^0 - 0 + C_1 = 1 \implies C_1 = 1 \\
  0^2 + C_2 = 0 \implies C_2 = 0
\end{align}

\subsection*{Longitud de curva}

Considere una cuerva C en $ \R^n $ se desea hallar la longitud de $ C $ tome un
conjunto de puntos en $ C $, $x_0, x_1, x_2, ..., x_n$. Tome ahora los segmentos
que unen los puntos $ x_k $ con $ x_{k + 1} $, $ k = 0, 1, .., n - 1 $

Sea $ r:I \subseteq \R \longrightarrow \R^{n} $ una paramtrización de $ C $, con
$ r(t_k) = x_k, k = 0 , ... , n$. Ahora $r(t_x) - r(t_{x - 1})$ es el vector que
va desde $x_{k - 1}$ hasta $x_{k}$

La longitud de $ C $ es $ L = \sum_{k=1}^{n}||r(t_x) - r(t_{x - 1})||$

Cuando $ x_k \longrightarrow x_{k-1} $ entonces $r(t_x) - r(t_{x - 1}) $ tiende
a $ r'(t_x) + h $, con $ h = t_x-t_{x_1} = \triangle t $

Así:
\begin{align}
  L \approx \sum_{k=1}^{n}||r'(t_x) \triangle t|| \\
  L &= \lim_{x\rightarrow\infty} \sum_{k=1}^{n}||r'(t_x)|| \triangle t \\
  &= \int_{t_0}^{t_x} ||r'(t_x)|| \mathrm{d}x
\end{align}

\subsection*{Ejemplo}

El perímetro de una circunferencia de radio $a$ en $ L = 2*\pi*a $

Para resolver esto tenemos que:

\begin{align}
  r(t) &= (a\cos(t), a\sin(t)), 0 \leq t \leq 2\pi \\
  r'(t) &= (-a\sin(t), a\cos(t)) \\
  || r'(t) || &= \sqrt{(-a\sin(t))^2 + (a\cos(t))^2)} \\
              &= \sqrt{a^2(\sin(t))^2 + a^2(\cos(t))^2} \\
              &= a
\end{align}

\subsection*{Ejemplo}

Halle la longitud de la curva dad por
$ r(t) = i + t^2j + t^3k, 0 \leq t \leq 1 $ por la que la solución dada es:
\begin{align}
  r(t) &= (1, t^2, t^3), 0 \leq t \leq 1 \\
  r'(t) &= (0, 2t, 3t^2) \\
  || r'(t) || &= \sqrt{0^2 + (2t)^2 + (3t^2)^2)} \\
              &= \sqrt{4t^2 + 9t^4} \\
  L &= \int_0^1 \sqrt{4t^2 + 9t^4} \mathrm{d}t \\
    &= \int_0^1 \sqrt{4 + 9t^2} \mathrm{d}t \\
    &= \int_0^1 \sqrt{u} \mathrm{d}t, u = 4 + 9t^2 \implies \mathrm{d}t = \frac{\mathrm{d}u}{18t} \\
    &= \frac{2}{18} \frac{\sqrt{u}*u}{3}  \\
    &= \frac{(4 + 9t^2)^{\frac{3}{2}}}{27} \Biggr|_{t=0}^{t=1} \\
    &= \frac{-8}{27}  \\
\end{align}

\section{}
\end{document}
